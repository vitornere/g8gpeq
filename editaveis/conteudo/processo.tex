\chapter[Processo]{Processo}

\section{Dados de Entrada}

De acordo com \cite{slack}, um modelo de transformação é composto por input, processo de transformação e output. Os inputs são os recursos de entrada geralmente classificados recursos a serem transformados e recursos de transformação. Os recursos a serem transformados são materiais, informações e consumidores. Os recursos de transformação são compostos por instalações (prédios, equipamentos, tecnologia) e funcionários (pessoas que operam as instalações) que agem sobre os recursos transformados.
 
Em uma organização, o sistema de produção pode agir em macrooperações ou em microoperações. A macrooperação se refere à produção principal de uma empresa, enquanto as microoperações se referem às produções menores que alimentam e sustentam a macrooperação. Por exemplo, uma empresa de propaganda tem sua macrooperação a campanha de divulgação de uma empresa específica que depende de microoperações como a criação do texto, o trabalho das imagens para veiculação da propaganda, a produção da mídia escolhida para a campanha.

\cite{slack} classifica as operações de produção segundo volume de output, variedade de output, variação da demanda do output e grau de contato com o consumidor envolvido na produção do output. 

Pode-se definir a produção em três termos: função produção, gerentes de produção e administração da produção. A função produção se encarrega de reunir os recursos para a produção de bens e serviços. Os gerentes de produção se encarregam de controlar os recursos envolvidos pela função produção. Administração da produção é a ferramenta do gerente de produção para gerir a função produção de maneira eficiente.

Ao final do processo de transformação dos inputs são criados os bens ou serviços (output). Esses produtos serão comercializados pela empresa para garantir sua sustentabilidade e crescimento.

\subsection{Recursos Transformados}

Com base nessa definição de \cite{slack}, e tido como inputs da pastelaria Bom Sabor:
Os seguintes recursos transformados (recursos convertidos para a produção do pastel):

\begin{itemize}
\item Farinha;
\item Óleo ou gordura vegetal;
\item Água e;
\item Antimofo;
\item Recheio.
\end{itemize}

\subsection{Recursos Transformadores}

E recursos transformadores(agem sobre os recursos transformados):

\begin{itemize}
\item \textbf{Máquinas:}
	\begin{itemize}
	\item Massiera;
	\item 7 rolo - Manual;
	\item Cilindro;
	\item Fechador Automático.
	\end{itemize}
\item \textbf{Receita do produto (quantidade das matérias primas.)}
\item \textbf{Funcionários:} 
	\begin{itemize}
	\item 4 na produção do pastel;
	\item 2 na parte administrativa;
	\item 2 como Motorista.
	\end{itemize}
\end{itemize}