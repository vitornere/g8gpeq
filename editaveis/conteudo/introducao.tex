\chapter[Introdução]{Introdução}

No presente trabalho será descrito e trabalhado o processo produtivo da fabrica de pasteis bom sabor. Para isso temos que compreender a importância de um processo bem estruturado, primeiramente, processo produtivo é a utilização de recursos (materiais, pessoas, financeiros e informação), para o alcance dos objetivos da organização, por isso, para obtermos uma produção de qualidade e que atinja seus objetivos precisamos analisar seu processo e estudá-lo como faremos nesse trabalho.

\section{Objetivo}

O trabalho tem como objetivo a análise do processo produtivo da fabrica de pastéis Bom Sabor, e além dessa, uma proposta de melhoria no processo produtivo que seja obtida com base nos estudos e aprendizados nas aulas de Gestão da Produção e Qualidade, ministradas pela professora Rejane Maria da Costa e nos textos e vídeos postados pela mesma.

\section{Empresa}

A fábrica de pastéis Bom Sabor tem uma historia muito interessante, seu dono possuía o sonho de ter uma empresa. Esse sonho começou quando ele pediu emprestado a um amigo uma pequena quantia para comprar e revender alguns poucos salgados. Com essa compra e venda dando resultados, ele decidiu produzir seus próprios salgados, foi quando percebeu seu sonho se aproximando. Após algum tempo de trabalho árduo, ele viu uma oportunidade de alugar um galpão e então tentar seguir seu sonho. Hoje a fabrica é compreendida por dois galpões e atende o mercado consumidor até em barreiras. Orgulhosamente o dono da fabrica a apresentou para nós e contou sua historia. História essa não apenas de a abertuda de uma empresa, mas da realização de um sonho.