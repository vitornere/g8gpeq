\chapter[Introdução]{Introdução}

No presente trabalho será descrito e trabalhado o processo produtivo da fábrica de pasteis bom sabor. Para isso se torna necessário compreender a importância de um processo bem estruturado. Primeiramente, processo produtivo é a utilização de recursos (materiais, pessoas, financeiros e informação) para o alcance dos objetivos da organização, por isso, para obtermos uma produção de qualidade e que atinja seus objetivos é preciso analisar seu processo e estudá-lo como será feito nesse trabalho.

\section{Objetivo}

O trabalho tem como objetivo a análise do processo produtivo da fábrica de pastéis Bom Sabor, e além dessa, uma proposta de melhoria no processo produtivo que seja obtida com base nos estudos e aprendizados nas aulas de Gestão da Produção e Qualidade, ministradas pela Prof$^{a}$. Dr$^{a}$. Rejane M. da C. Figueiredo e nos textos, livros e vídeos postados pela mesma.

\section{Empresa}

A fábrica de pastéis Bom Sabor tem uma história muito interessante, seu dono possuía o sonho de ter uma empresa. Esse sonho começou quando o mesmo pediu emprestado a um amigo, uma pequena quantia para comprar e revender alguns poucos salgados. Com essa compra e venda dando resultados, ele decidiu produzir seus próprios salgados, foi quando percebeu seu sonho se aproximando. Após algum tempo de trabalho árduo, ele viu uma oportunidade de alugar um galpão e então tentar seguir seu sonho. Hoje a fábrica é composta por dois galpões e atende o mercado consumidor até em Barreiras-BA. Orgulhosamente, o dono da fábrica apresentou a mesma para nós e contou sua história. Relato esse não apenas de abertura de uma empresa, mas da realização de um sonho.