\section[Estratégia de Produção]{Estratégia de Produção}

A estratégia da produção é o padrão de ações e decisões que  define o arranjo da empresa em ambiente, com o objetivo de alcançar os objetivos de longo prazo. A estratégia é a definição e o processo especifico que determinam as atividades, os objetivos e as funções da produção. Nisso tem-se a determinação de qual método ser usado, quais ações e decisões serão tomadas, qual modelo e procedimento adotado. Dessa forma pode-se  ter uma previsão sobre a produção, podendo prever futuros prejuízos possíveis até melhoramento do processo de produção, diminuindo custo, tempo e aumentando a qualidade.

A empresa Bom Sabor tem algumas estratégias de produção: produzem pastel somente após ter feito o pedido, para não ter que ocupar o estoque por muito tempo, evitar que os pasteis estraguem e evitar super produção. Os ingredientes do recheio que tem pouco tempo de validade são comprados constantemente e em pouca quantidade, para ter sempre ingredientes frescos, já os que têm maior validade são comprados em grandes quantidades.

A massa do pastel e seus ingredientes têm maior validade, são mais difícil  de estragar, sendo assim têm grande quantidades em estoque, e também a massa é uma das principais partes do pastel, sendo  constantemente  usado em grande quantidade.
Dessa forma é possível prever e evitar alguns prejuízos com perda de produto. 

